\SetPicSubDir{ch-Intro}

\chapter{Introduction}
\label{ch:Introduction}


\section{Background}
\label{sec:Intro:thesis_background}

Nowadays, mobile computing devices, such as phones, have become everyday device companions \cite{hakoama2011impact, dey_getting_2011}. With the increase in mobile device platforms and digital services, the number of daily notifications that users receive has increased \cite{chuang_ambient_2017, noauthor_push_2019, statista_us_2021}. Typically, a mobile device user receives more than 60 notifications per day \cite{sahami_shirazi_large_scale_2014, pielot_dismissed_2018}, while a tech-savvy user, such as a college student, receives more than 400 mobile notifications per day \cite{lee_hooked_2014}. Moreover, most notifications are attended to within a few minutes \cite{pielot_situ_2014, sahami_shirazi_large_scale_2014, pielot_dismissed_2018}.

Notifications are the cues presented via different modalities (e.g., visual cues, auditory signals, haptic alerts) that are generated by an application or service, which relays information to a user outside their current focus of attention \cite{iqbal_notifications_2010}. They help guide users' attention and proactively deliver additional information in a timely manner \cite{horvitz_balancing_2005, paul_interruptive_2015, iqbal_notifications_2010, pielot_situ_2014}.

However, notifications are the most common form of interruption in information technology and computer-related systems \cite{addas_many_2015}. They cause unwanted effects on users while they are engaged in other tasks, including reducing task performance, increasing mental load, generating negative emotions, and disrupting social interactions \cite{stothart_attentional_2015, bailey_effects_2001, cutrell_notification_2001, adamczyk_if_2004, kushlev_silence_2016, mcatamney_examination_2006, horvitz_busybody_2004, leiva_back_2012}. For example, users may be alerted with a notification while walking and slow down to check the notification before resuming their walking at normal speed. Meanwhile, completely disregarding or ignoring notifications is not feasible because it can cause anxiety and encourage self-interruptions due to the fear of missing out on important information \cite{pielot_productive_2017, iqbal_notifications_2010}.

Given that notifications have both positive and negative aspects, it is essential to find ways to manage them properly; and with the increase in notifications on mobile devices, minimizing their negative consequences has become a necessity.





\section{Motivation}
\label{sec:Intro:thesis_motivation}

Optical see-through head-mounted displays (OST HMDs, OHMDs) or augmented reality smart glasses (ARSG) are a trending mobile platform where virtual information (e.g., computer-generated graphics) is superimposed on the semi-transparent near-eye display, supporting always-on access to information \cite{itoh_towards_2021, azuma_survey_1997, rauschnabel_augmented_2015}. Their ability to provide digital content while maintaining situational awareness of the physical background \cite{orlosky_managing_2014, gruenefeld_guiding_2018}, as well as the ability to superimpose digital content on physical objects \cite{azuma_survey_1997}, have attracted attention in research as well as the industry \cite{rauschnabel_augmented_2015}. They have been used in different areas, including but not limited to medical/healthcare \cite{mitrasinovic_clinical_2015, chicchi_giglioli_augmented_2015}, manufacturing and maintenance \cite{wang_comprehensive_2016, peng_roma_2018}, social \cite{piumsomboon_mini_me_2018, ofek_reducing_2013, mcatamney_examination_2006}, military \cite{bach_us_2021}, navigation \cite{roy_follow_my_lead_2017, anandapadmanaban_holo_sextant_2018}, culture and tourism \cite{de_paolis_natural_2014}, education \cite{ibrahim_arbis_2018, akcayir_advantages_2017, zarraonandia_augmented_2013}, and sports \cite{fan_augmented_2016, ban_augmented_2013, piekarski_arquake_2002}.

With the increase in the number of OHMDs in the \textit{consumer} market \cite{alsop_ar_2022} (e.g., Vuzix\footnote{\url{https://www.vuzix.com/}}, Nreal\footnote{\url{https://www.nreal.ai/}}, Xiaomi\footnote{\url{https://www.xda-developers.com/xiaomi-ar-smart-glass-hands-on/}}, Lenovo\footnote{\url{https://www.lenovo.com/us/en/thinkrealitya3/}}, Snap\footnote{\url{https://www.spectacles.com/new-spectacles/}}, Oppo\footnote{\url{https://www.oppo.com/en/newsroom/press/oppo-air-glass/}}, TCL\footnote{\url{https://www.rayneo.com/}}, Apple in 2023\footnote{\url{https://www.tomsguide.com/news/apple-glasses}}), it is predicted to be the next everyday computing companion of humans and can potentially supplant mobile phones \cite{azuma_road_2019, chia_smart_2019, conditt_worlds_2018}.


\begin{figure}[hptb]
  \centering
  \includegraphics[width=\linewidth]{\Pic{Heads-Up_evolution.pdf}}
  \caption[Heads-up computing evolution]{Heads-up computing evolution. Source: \cite{zhao_headsup_2023}.}
  \label{fig:Intro:heads_up}	  
\end{figure}

OHMD is the leading device platform that enables the emergent heads-up computing interaction paradigm \cite{zhao_headsup_2023, nus_hci_lab_heads_up_2022} (\autoref{fig:Intro:heads_up}), which can potentially mitigate the harmful aspects of the prevalent mobile interaction paradigm (which focuses on mobile phones, tablets), such as "smartphone zombies" who lack situational awareness \cite{appel_smartphone_2019} and health issues such as "text neck" due to heads-down posture \cite{shin_effects_2014, gustafsson_texting_2017}. Moreover, OHMDs enable increased situational awareness \cite{orlosky_managing_2014, gruenefeld_guiding_2018, lucero_notifeye_2014}, allowing increased multitasking with digital information \cite{chaturvedi_peripheral_2019, orlosky_managing_2014}.


However, presenting notifications on OHMD can divert users' attention and cause more interruption from visual stimuli than presenting notifications on other mobile devices, such as phones, which may not always be present in users' visual field \cite{mcatamney_examination_2006, sabelman_real_life_2015}.

Thus, OHMD notifications need to be designed to support heads-up computing with minimal adverse effects. Although there have been many research studies on notification design and management that focus on presenting desired information to users in an efficient manner while minimizing the disruption for desktop and mobile computing \cite{anderson_survey_2018, horvitz_balancing_2005, ho_using_2005, pejovic_interruptme_2014}, the difference between form factor and affordances of OHMDs compared to other device platforms (e.g., mobile phones, \autoref{tab:intro:ohmd_phone_affordance}) \cite{janaka_glassmessaging_2023, stephanidis_properties_2015, zhu_bishare_2020, vadas_reading_2006, zhou_ubiquitous_2019} makes it harder to adopt the existing guidelines \textit{directly} for heads-up computing (e.g., \cite{rzayev_effects_2020, costanza_eye_q_2006}). Even though several studies focused on OHMD notification design (e.g., \cite{ofek_reducing_2013, chaturvedi_peripheral_2019}), they lack the exploration of maintaining communicative effectiveness (see \autoref{ch:Relatedwork} for details).

With the literature gap (\autoref{ch:Relatedwork}) and the increase of OHMD devices in the consumer market \cite{alsop_ar_2022}, we see the need and opportunities to manage heads-up OHMD notifications that motivate this thesis.



\begin{table}[htbp]
\centering
\caption[Affordances of OHMDs vs Phones]{A Summary of different affordances supported by OHMDs and Phones \cite{janaka_glassmessaging_2023, stephanidis_properties_2015, zhu_bishare_2020, vadas_reading_2006, zhou_ubiquitous_2019}. Note: This list is not exhaustive. }
\begin{tabular}{@{}lll@{}}
\toprule
\textbf{Aspect} & \textbf{OHMD} &\textbf{Phone} \\ \midrule
\multirow{2}{*}{Display} & See-through & Opaque \\
 & Near-eye & Distant \\ 
 & Central and peripheral vision & Central vision  \\ \midrule
Access & Always-on \& On-demand & On-demand \\ \midrule
Viewing & Heads-up & Heads-down \\ \midrule
Wearing & Wearable & Holdable \\ \midrule
Hands-busy & Hands-free & Hands-busy \\ \midrule
Content privacy & Only be seen by the wearer &  Can be seen by bystanders \\
\bottomrule
\end{tabular}
\label{tab:intro:ohmd_phone_affordance}
\end{table}



\section{Thesis scope: visual OHMD notifications}
\label{sec:Intro:thesis_scope}

Humans intake more than 80\% of all information via the visual modality (i.e., through their eyes) \cite{rosenblum2011see}. In the context of notifications, which can be presented in multiple modalities, the visual modality presents richer information with higher encoding capability and can also be combined with other modalities in a complementary manner \cite{hsia_information_1971, rau_modality_2019}. Moreover, OHMDs use the visual modality as the primary output modality \cite{itoh_towards_2021, milgram_taxonomy_1994, azuma_survey_1997}. Therefore, throughout this thesis, we focused on visual notifications (i.e., notifications presented only in the visual modality) on OHMDs.


In addition, people usually receive notifications when engaged in other activities \cite{fischer_effects_2010, mehrotra_my_2016, yuan_how_2017, pielot_situ_2014}. Accordingly, we define the \textit{primary task} as the main activity that users are engaged with and the \textit{secondary tasks} as the new activities that users will engage in when they receive notifications. For example, when an OHMD user navigates on a busy road, the user may receive a notification and attend to it; hence, the \textit{primary} task is the navigation, and attending to the notification becomes the \textit{secondary} task. 

When users attend to notifications, part of their focused (visual) attention is diverted from the primary task to the notifications. Furthermore, users feel distracted or interrupted from their primary tasks if such attention diversion results in a cost, such as when primary task performance is negatively affected by the notification \cite{mccrickard_attuning_2003}.

As shown in \autoref{fig:Intro:attention_utility_tradeoff}, the perceived cost of interruption depends on \textit{situational factors}, such as context, user characteristics, and information characteristics, as well as \textit{user goals} reflected in their expectation of interruption, reaction, and comprehension \cite{mccrickard_attuning_2003, mccrickard_model_2003}. For example, the user may be willing to accept a higher interruption if the notification reminds them of a crucial upcoming meeting that starts in a few minutes (information characteristics), even if the user is writing an important report (context), as the upcoming meeting may require an urgent reaction (e.g., rushing to the meeting location). At the same time, if the above notification is about a close friend's birthday next week, the user may not willingly receive interruptions but want to remember it later (i.e., needing higher comprehension). So there is a trade-off between the attention cost and the utility of notifications; thus, notification systems should minimize the attention cost while maximizing the utility of providing secondary information (i.e., additional information outside the current focus of attention) \cite{mccrickard_attuning_2003, mccrickard_model_2003, gluck_matching_2007}.

\begin{figure}[hptb]
  \centering
  \includegraphics[width=0.9\linewidth]{\Pic{attention_utility_tradeoff.png}}
  \caption[Attention benefits and costs.]{Attention benefits and costs. Users expect to gain benefits associated with fulfilling users' goals (left side) by sacrificing attention from other tasks. Costs can be exacerbated by factors of the current situation (right side). Source: McCrickard et al. \cite{mccrickard_attuning_2003}.}
  \label{fig:Intro:attention_utility_tradeoff}	  
\end{figure}


Visual notifications directly affect visual attention as information is directly presented in the near-eye displays of OHMDs. Therefore, in this thesis, we focused on \textit{information characteristics}, particularly \textit{visual information presentation}, which directly affects visual attention \cite[Ch1]{styles_psychology_2005, ware_information_2013}, to explore ways of designing OHMD notifications further.

Given that visual information intake depends on \textit{human visual perception} \cite{goldstein_sensation_2016, goldstein_blackwell_2005, kalat_biological_2012}, we hypothesized that human vision and visual perception (see \autoref{sec:Relatedwork:notification_evaluation} for details) could be utilized to \textbf{design} the presentation format of OHMD notifications. Thus, we explored using visual perception properties in designing OHMD notifications.



McCrickard et al. \cite{mccrickard_model_2003, mcatamney_examination_2006, mccrickard_attuning_2003, mccrickard_establishing_2003} developed a conceptual model, the \textit{IRC framework}, based on the attention-utility trade-off, focusing on user goals to improve design decisions for notification systems. As shown in \autoref{fig:Intro:IRC_framework}, their descriptive and prescriptive model uses three critical parameters: \textit{Interruption} (I), \textit{Reaction} (R), and \textit{Comprehension} (C) (see \autoref{sec:Relatedwork:notification_evaluation} for details). Given that this model can identify the differences between the targeted design model (i.e., expected parameters) and the actual user's model (i.e., resulting parameters), we use this model to \textbf{evaluate} our proposed notification designs.



\begin{figure}[hptb]
  \centering
  \includegraphics[width=0.9\linewidth]{\Pic{IRC_framework.png}}
  \caption[Notification systems categorizations based on IRC framework]{Notification systems categorizations according to the blend of design model objectives (representing user goals) of interruption (I), reaction (R), and comprehension (C) with low (0) or high (1) values. Source: McCrickard et al.  \cite{mccrickard_model_2003}.}
  \label{fig:Intro:IRC_framework}	  
\end{figure}


As the utility benefits and attention costs depend on the \textit{context} as well as \textit{information characteristics} \cite{mccrickard_attuning_2003, mccrickard_model_2003} (see \autoref{fig:Intro:attention_utility_tradeoff}), and given that notifications provide various types of secondary information \cite{sahami_shirazi_large_scale_2014, pielot_situ_2014}, our notification designs are focused on specific contexts and particular types of secondary information (refer to \autoref{ch:Progressbar}-\ref{ch:Gradnotif} for detailed information). Moreover, we chose the most suitable visual perception properties based on the usage context and secondary information type (see \autoref{sec:Intro:thesis_RQ} for details) to support our targeted design model.
We chose social interaction and work settings as usage scenarios primarily because the adverse effects of notifications in these settings can be quite severe \cite{sabelman_real_life_2015, mcatamney_examination_2006}.



\section{Research questions}
\label{sec:Intro:thesis_RQ}

This thesis aims to answer the high-level research question: \textit{How can we minimize the attention costs of notifications in heads-up computing?} Based on the scope of the research, the following thesis question will be addressed:
\begin{itemize}
    \item \RQMainThesis{}
\end{itemize}
To answer the above question, as illustrated in Figure \ref{fig:Intro:thesis_rq_overview}, we explore various ways to leverage visual perception properties (\autoref{sec:Relatedwork:human_visual_perception}) in the \textit{design} of OHMD notifications, with the goal of minimizing unwanted distractions. Visual perception is influenced by factors such as the information receiver (i.e., the eyes), the information source (e.g., virtual content on OHMD), and the information channel (i.e., light) \cite[Ch~6]{ware_information_2013, kalat_biological_2012}. Therefore, we focus on three fundamental aspects that affect each of these elements: vision region (receiver), form/pattern (source), and luminance (channel).


\begin{figure}[hptb]
  \centering
  \includegraphics[width=1\linewidth]{\Pic{thesis_rq_overview.pdf}}
  \caption[The research question structure]{The research question structure shows how the visual perception properties are used to answer the selected research questions in the scope of visual OHMD notifications.}
  \label{fig:Intro:thesis_rq_overview}	  
\end{figure}

\vspace*{2mm}
In the first project (\autoref{ch:Progressbar}), we explored the utilization of different regions of visual perception, particularly paracentral and near-peripheral vision \cite{strasburger_peripheral_2011}, to distribute OHMD notification contents in order to reduce the information load in central vision. The research question guiding this investigation was:
\begin{itemize}
    \item \RQMainProgressBar{}
\end{itemize}
To address this question, we developed a circular progress bar, an OHMD progress notification displayed in the paracentral and near-peripheral vision areas to convey time availability and task completion information during social interactions.

\vspace*{4mm}
In the second project (\autoref{ch:Iconnotif}), we explored ways to utilize a \i{form} perception property, specifically the fact that shapes are generally easier to recognize than text  \cite{tijus_design_2007}, for OHMD notifications. The research question guiding this project was:
\begin{itemize}
    \item \RQMainIconNotif{}
\end{itemize}
To address this question, we examined the use of icons to represent text notifications in a work setting, which had received limited attention in previous research.

\vspace*{4mm}
In the third project (\autoref{ch:Gradnotif}), we investigated how luminance adjustments \cite[Ch~3]{ware_information_2013} can be employed to minimize the disruptive impact of sudden visual stimuli from OHMD notifications. The research question guiding this project was:
\begin{itemize}
\item \RQMainGradNotif{}
\end{itemize}
To address this question, we designed fading text notifications, where the light intensity of the notification changes gradually. We aimed to determine the optimal fading duration for a work setting. 

\vspace*{4mm}
To address each research question, we conducted controlled studies, specifically randomized controlled trials, with the approval of our university's institutional review board (IRB). The specific details of these studies are provided in each respective chapter.




\section{Thesis contribution}
\label{sec:Intro:thesis_contribution}

This thesis makes several contributions to the field of visual OHMD notifications and their impact on attention costs in heads-up computing. These contributions are outlined as follows:

First, the thesis explores the utilization of different regions of visual perception, specifically paracentral and near-peripheral vision, for distributing OHMD notification contents during social interactions. Novel OHMD progress notifications were designed to leverage paracentral and near-peripheral vision, and their effectiveness was empirically evaluated in simulated and realistic settings. Through this evaluation, the thesis investigates the trade-offs between notification design and the quality of social interactions. Additionally, potential OHMD designs utilizing paracentral and near-peripheral vision for other multitasking scenarios are proposed based on the findings.

Second, the thesis introduces a method for leveraging the visual perception property that shapes are generally easier to recognize than text in order to minimize the interruption caused by OHMD notifications. It demonstrates the feasibility and desirability of transforming text notifications into pictorial notifications for OHMDs during multitasking. Factors influencing the effectiveness of pictorial notifications in the OHMD context are examined, and each type of notification is empirically evaluated through comparative studies. The thesis further extends its findings to a realistic setting using an ecological study, providing insights into the trade-offs associated with using text and pictorial notifications. Ultimately, it demonstrates that pictorial notifications effectively reduce unwanted interruptions from frequently received short notifications in both stationary and mobile multitasking scenarios.


Third, the thesis investigates methods of controlling luminance to minimize the interruption caused by sudden visual stimuli of OHMD notifications. The use of fading animation is explored and compared to commonly used animations to understand the impact of luminance control. The findings indicate that fading animations can effectively reduce the interruption caused by OHMD notifications, although various factors need to be considered when implementing fading techniques.


In conclusion, the thesis provides a comprehensive discussion on the generalization of the results and their implications for the design of future heads-up visual OHMD notifications and information presentations on near-eye displays. It also outlines potential future research directions for managing interruptions in heads-up computing. These include leveraging visual perception properties, combining modalities, evaluating OHMD notifications, and exploring new application areas for OHMDs.


Overall, this thesis contributes to the field's knowledge regarding notification designs for OHMDs, as well as the evaluation and management of interruptions in heads-up computing. By utilizing the inherent properties of near-eye displays, it pioneers the redesign of notifications for heads-up computing, aiming to facilitate effective access to information anytime and anywhere using OHMDs.






\section{Thesis structure}
\label{sec:Intro:thesis_structure}

As shown in \autoref{fig:Intro:thesis_structure}, the thesis is structured into three main chapters, each presenting different notification designs aimed at minimizing the negative aspects of visual OHMD notifications. These designs are guided by human visual perception properties and evaluated using the IRC framework to assess attention-utility trade-offs. The structure of the thesis is as follows:

\begin{figure}[hptb]
  \centering
  \includegraphics[width=0.95\linewidth]{\Pic{thesis_structure.pdf}}
  \caption[The thesis structure]{The thesis structure illustrates the interconnectedness of the chapters. Visual perception properties are employed in designing OHMD notifications, while the IRC framework is utilized to evaluate their effectiveness.}
  \label{fig:Intro:thesis_structure}	  
\end{figure}

\begin{itemize}
    \item \autoref{ch:Relatedwork} situates the research presented in this thesis within the broader body of research on visual perception, interruptions, and notifications.
    
    \item \autoref{ch:Progressbar} explores the utilization of paracentral and near-peripheral vision for presenting OHMD notifications. It includes comparative studies that evaluate the proposed design for progress notifications against existing designs in simulated and realistic social interaction settings. The chapter also discusses guidelines and limitations regarding the utilization of paracentral and near-peripheral vision for information presentation on OHMDs.
    
    \item \autoref{ch:Iconnotif} focuses on establishing the feasibility and desirability of transforming text notifications into pictorial notifications in the OHMD context. It begins by identifying the advantages of pictorial notifications and then examines the conditions under which pictorial notifications are more effective than text notifications. The chapter concludes with a comparative study conducted in a realistic setting to validate the findings from lab settings and discuss the trade-offs associated with transforming text into pictorial representations.
    
    \item \autoref{ch:Gradnotif} presents a method for utilizing luminance contrast to minimize unwanted attention attraction caused by OHMD notifications. By comparing the proposed fading animation with existing notification animations, this chapter identifies the conditions in which fading animation is more effective in both stationary and mobile situations. Furthermore, it discusses the generalization of results across different tasks and explores the trade-offs associated with using OHMD fading animations.
    
    \item Finally, the thesis concludes with \autoref{ch:Conclusion}, which provides a comprehensive review of the thesis, including design implications, limitations, and suggestions for future work.
    
\end{itemize}




\section{Publications during Ph.D.}
\label{sec:Intro:publications}

\begin{refsection}
\nocite{%
    janaka_can_2023,
    janaka_notifade_2023,
    janaka_glassmessaging_2023,
    runze_paraglassmenu_2023,
    zhang_adaptReview_2023,
    janaka_paracentral_2022,
    janaka_visual_2022,
    ghosh_eyeditor_2020,  
}

\defbibnote{PubListPrenote}{%
% provide some text, if any, before the list of publications
In reverse chronological order:
}
\defbibnote{PubListPostnote}{%
% provide some text, if any, after the list of publications
}

\bookmarksetup{startatroot}
\newrefcontext[sorting=none]
\printbibliography[
  % heading=bibintoc,
  heading=none,
  title={Publications during Ph.D.},
  prenote=PubListPrenote,
  postnote=PubListPostnote
]
\end{refsection}
 % optional to include your publication list
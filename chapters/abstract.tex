\begin{abstract}


Despite their ability to provide crucial secondary information, notifications can interrupt users and interfere with their primary tasks. Hence, with the increased usage of mobile devices and digital services, mitigating the adverse effects of notifications has become an important research challenge in the Human-Computer Interaction (HCI) field. Optical See-Through Head-Mounted Display (OST HMD, OHMD, a.k.a., augmented-reality smart glasses) is an emergent mobile and wearable device platform that can potentially supplant the mobile phone and become an everyday device companion.

This thesis explores several ways of minimizing the attention costs of OHMD notifications based on human visual perception. To achieve these objectives, it is proposed to: 1) utilize paracentral and near-peripheral regions, 2) transform text content to graphical format, and 3) change the luminance of notification content, when displaying OHMD notifications.
Then, each proposed OHMD notification design was evaluated with a series of user studies. Our results contribute to creating attention-maintaining visualizations for OHMDs, suggesting that by modifying the information presentation of OHMD notifications, users can obtain secondary information with less distraction to primary tasks than the existing presentation techniques.

Finally, we discuss the design implications and how our research addresses the existing challenges in the heads-up computing paradigm.


\end{abstract}
